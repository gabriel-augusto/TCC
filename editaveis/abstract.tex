\begin{resumo}[Abstract]
 \begin{otherlanguage*}{english}

An acoustic design should be carefully studied because aspects, such as dimensions and materials, applied have a direct influency at the sound of the environment. The control of the sound in an environment, where the dimensions have been determined, is accomplished through the application of materials which influence the sound. The use of these materials should be carefully studied in order to apply materials that contribute to a good acoustic design and avoiding the use of unnecessary materials. The use of computer simulations has been increasingly common in everyday acoustic designer in order to achieve more efficient designs that meet the acoustical requirements expected by the client. In this work, the design of a simulator that allows sound behavior simulation within a closed environment will be presented. The simulation consists of a sound representation as a particle that travels in a certain environment, identifying collision points and generating the respective sound absorption and reflections in order to simulate the reverberation event in the environment. This project aims to assist acoustic experts and / or designers, so they can monitor and evaluate if the parameters of the room are really suitable. It is a free software tool based on multi-agent paradigm, using some features of object orientation.

   \vspace{\onelineskip}
 
   \noindent 
   \textbf{Key-words}: acoustic simulator. reflection. absorption. reverberation. multi-agents. 
 \end{otherlanguage*}
\end{resumo}
