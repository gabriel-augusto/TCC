% Define cores em RGB
\definecolor{dkgreen}{rgb}{0,0.6,0}
\definecolor{gray}{rgb}{0.5,0.5,0.5}
\definecolor{mauve}{rgb}{0.58,0,0.82}
 
% Configuração para exibir código e C++
\lstset{
language = Java, % Linguagem de programação
basicstyle = \footnotesize, % Tamanho da fonte do código
numbers = left, % Posição da numeração das linhas
numberstyle = \tiny\color{blue}, % Estilo da numeração de linhas
stepnumber = 1, % Numeração das linhas ocorre a cada quantas linhas?
numbersep = 10pt, % Distância entre a numeração das linhas e o código
backgroundcolor = \color{white}, % Cor de fundo
showspaces = false, % Exibe espaços com um sublinhado
showstringspaces = false, % Sublinha espaços em Strings
showtabs = false, % Exibe tabulação com um sublinhado
frame = trBL, % Envolve o código com uma moldura, pode ser single ou trBL
rulecolor = \color{black}, % Cor da moldura
tabsize = 2, % Configura tabulação em x espaços
captionpos = b, % Posição do título pode ser t (top) ou b (bottom)
breaklines = true, % Configura quebra de linha automática
breakatwhitespace= false, % Configura quebra de linha
title = \lstname, % Exibe o nome do arquivo incluido
%caption = \lstname, % Também é possível usar caption no lugar de title
keywordstyle = \color{blue}, % Estilo das palavras chaves
commentstyle = \color{dkgreen}, % Estilo dos Comentários
stringstyle = \color{mauve}, % Estilo de Strings
escapeinside = {\%*}{*)}, % Permite adicionar comandos LaTeX dentro do seu código
morekeywords     ={*,...} % Se quiser adicionar mais palavras-chave
}

\begin{apendicesenv}

\partapendices
\begin{comment}
\chapter{Documento de Arquitetura}

\section{Introdução}
Este documento apresenta a arquitetura proposta para o . A arquitetura é apresentada através de um conjunto de visões que juntas visam cobrir os principais aspectos técnicos relativos ao desenvolvimento e implantação do sistema em questão. O objetivo é capturar e formalizar as principais decisões tomadas com relação à arquitetura do sistema.

Neste documento é detalhado as principais partes da arquitetura implementada para a Ferramenta de Software desenvolvida. A arquitetura é composta por padrões de projetos orientados a objetos e por agentes comportamentais de Software. O Padrão Arquitetural adotado é o Model-View-Controller adaptado ao uso do paradigma Orientado a Agentes Comportamentais. Este documento contém ainda, uma proposta de refatoração
arquitetural no tocante aos Agentes implementados, tais sugestões foram obtidas através da experiência do desenvolvedor com o paradigma e como ele é interpretado pela plataforma JADE.

...
\end{comment}
% ------------------------------------------ CÓDIGO ------------------------------------- %

\chapter{Código fonte da classe Ambient}

\lstinputlisting[language=java]{src/simulator/agents/Ambient.java}

\chapter{Código fonte da classe SoundSource}

\lstinputlisting[language=java]{src/simulator/agents/SoundSource.java}


\chapter{Código fonte da classe Sound}

\lstinputlisting[language=java]{src/simulator/agents/Sound.java}

\chapter{Código fonte da classe Obstacle}

\lstinputlisting[language=java]{src/simulator/objects/Obstacle.java}


\chapter{Código fonte da classe Line}

\lstinputlisting[language=java]{src/simulator/objects/Line.java}


\chapter{Código fonte da classe NormalLine}

\lstinputlisting[language=java]{src/simulator/objects/NormalLine.java}


\chapter{Código fonte da classe VerticalLine}

\lstinputlisting[language=java]{src/simulator/objects/VerticalLine.java}

\chapter{Código fonte da classe Location}

\lstinputlisting[language=java]{src/simulator/objects/Location.java}

\chapter{Código fonte da classe SoundObject}

\lstinputlisting[language=java]{src/simulator/objects/SoundObject.java}

\chapter{Código fonte da classe SoundSourceObject}

\lstinputlisting[language=java]{src/simulator/objects/SoundSourceObject.java}

\chapter{Código fonte da classe AmbientObject}

\lstinputlisting[language=java]{src/simulator/objects/AmbientObject.java}

\end{apendicesenv}
