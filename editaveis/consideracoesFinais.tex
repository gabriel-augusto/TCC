\chapter{Conclusão}

Este trabalho teve como objetivo o desenvolvimento do simulador acústico que, por sua vez, é uma ferramenta multiparadigma baseada nos paradigmas multiagentes e orientado a objetos. Esta ferramenta consiste em simular o comportamento do som em um ambiente fechado e prover o tempo de reverberação (RT60) ao final de cada simulação. Dessa forma, é possível que o usuário informe parâmetros como dimensões do ambiente, obstáculos e suas posições, coeficientes de absorção sonora de cada obstáculo e, por fim, fontes sonoras e suas devidas configurações, tais como, posição, ângulo de abertura e direção. 

A partir da configuração desses itens, o usuário é capaz de executar e acompanhar uma simulação, podendo pausar, retomar ou encerrar a qualquer momento se assim desejar. Ao fim, o sistema informa o tempo de reverberação do ambiente com base nos parâmetros informados anteriormente pelo usuário. Neste trabalho, foi apresentado como o simulador acústico pode ser utilizado, desde a sua inicialização, inserções de obstáculos e fontes sonoras, até a realização da simulação de um ambiente acústico e a visualização do tempo de reverberação.

Para atingir o objetivo geral deste trabalho, foram definidos alguns objetivos específicos. A princípio, foi realizado um estudo a respeito dos conhecimentos de acústica, onde foi dada uma ênfase maior em ambientes acústicos e como se dava o comportamento do som dentro desses ambientes, a fim de identificar as variáveis a serem consideradas dentro do sistema, com o intuito de alcançar o primeiro objetivo específico deste trabalho, ou seja, estudar o comportamento do som dentro de ambientes acústicos, identificando variáveis acústicas presentes dentro desses ambientes.

Juntamente com o estudo sobre acústica, foi realizado também um estudo a respeito dos coeficientes de absorção dos materiais presentes no ambiente e como o som interagia com os mesmos, a fim de atender ao segundo objetivo específico deste trabalho. Desta forma, foi possível propor um suporte tecnológico adequado para a implementação do simulador, o qual foi o terceiro objetivo específico.

Durante a implementação da solução, foram exploradas técnicas de programação, bem como padrões de projeto e as demais boas práticas da engenharia de software, visando o desenvolvimento de um simulador manutenível e extensível. Para garantir a qualidade do código, foram realizados testes automatizados juntamente com a análise estática do código fonte, onde foi possível observar o nível de cobertura do código fonte e algumas métricas importantes referente ao código desenvolvido. Também houve uma preocupação em realizar atividades de refatoração de forma sistemática durante o desenvolvimento do simulador. Todas essas atividades contribuíram para que os objetivos específicos 4 e 5 deste trabalho fossem atendidos.

Neste trabalho, foi proposta a seguinte questão de pesquisa: \textit{"É possível desenvolver um sistema que simule o comportamento do som dentro de um ambiente fechado utilizando uma abordagem multiagentes?"}. Com base nos resultados desse trabalho, oriundos dos objetivos específicos, pode-se concluir que a resposta à questão é: Sim, o simulador atendeu ao objetivo geral para o qual foi proposto.

A expectativa do autor é que o simulador  possa ser evoluído e utilizado ou incorporado em novos simuladores e/ou suportes. O intuito é auxiliar projetistas e/ou especialistas em acústica no que tange o acompanhamento e a avaliação dos parâmetros de seus projetos.

\section{Trabalhos futuros}

O simulador desenvolvido neste trabalho de conclusão de curso aborda uma máquina de raciocínio capaz de representar o comportamento do som dentro de um ambiente. Com o intuito de permitir que este simulador fosse utilizado a nível de usuário, foi desenvolvida uma interface gráfica, possibilitando configurar um ambiente, obstáculos e fontes sonoras e então, realizar uma simulação. No entanto, a biblioteca de gráficos JFreeChart, limitou bastante a visualização da simulação, pois não era possível representar os obstáculos e os sons simultaneamente.

Um dos pontos de melhoria, seria investigar uma alternativa ao JFreeChart que possibilite a representação de todos os elementos dentro do ambiente. É preferível que essa nova alternativa possua recursos para a representação do ambiente em três dimensões, uma vez que o simulador já começa a apresentar recursos tridimensionais, os quais foram implementados e não podem ser visualizados adequadamente via JFreeChart..

A informação dos índices de absorção de cada material para cada faixa de frequência ocorre de forma manual no sistema atualmente. Uma potencial evolução no simulador seria investigar materiais comumente utilizados em obras que envolvem tratamento acústico e identificar seus índices de absorção para que possam então ser cadastrados no sistema, permitindo ao usuário escolher apenas o material que se aplica a cada obstáculo como uma alternativa à inserção do coeficiente de absorção de forma manual.

Por fim, outro ponto de extensibilidade do simulador é a implantação de novos parâmetros a serem coletados durante a simulação e apresentados ao usuário final. Existem diversos parâmetros adicionais que podem ser obtidos por meio da simulação como por exemplo: clareza, brilho, equilíbrio, ruído e distorção \cite{figueiredo}.

A arquitetura do simulador foi elaborada de modo a garantir uma ferramenta manutenível e extensível. O código fonte está disponível para a comunidade de software livre no GitHub\footnote{\url{https://github.com/gabriel-augusto/AcousticSimulator}}, um site para compartilhamento de projetos que utilizam a ferramenta de controle de versionamento Git.