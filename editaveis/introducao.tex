\chapter[Introdução]{Introdução}

Segundo \citeonline[pág.~17]{silva}, uma boa acústica em um ambiente é consequência da aplicação, pelo arquiteto, dos princípios da Acústica Arquitetônica.

Um projeto acústico deve ser cuidadosamente estudado. O mesmo deve ser funcional, isto é, todos os detalhes deverão ter uma razão de ser, a fim de se evitar o emprego de materiais supérfluos. Nesse contexto, o uso de simulações computacionais torna-se cada vez mais frequente no dia a dia do projetista acústico, procurando garantir projetos mais eficientes e que atendam às necessidades acústicas esperadas pelo cliente. \cite{silva}.

\section{Contextualização}

Projetos de estúdios de gravação, salas de concerto, teatros, salas de aula ou locais que necessitem de qualidade acústica para realizar suas atividades requerem estudos sobre suas dimensões, volumetria, forma e composição de suas superfícies. É papel do projetista acústico definir esses parâmetros a fim de garantir os requisitos acústicos do projeto. \cite{ramires}.

Em alguns casos, o volume e a forma desses ambientes já se encontram definidos, restando, portanto, ao projetista, definir quais materiais irão compor as superfícies desta sala, a fim de se obter um ambiente que cumpra com sua função acústica \cite{ramires}.

Segundo \citeonline[pág.~6]{santos}, a adaptação do ambiente, de acordo com o uso em questão, é realizada a partir do direcionamento da energia sonora dentro do mesmo. Esta redistribuição pode ser tratada de diferentes maneiras, de acordo com o projeto ou com sua finalidade, seja ela música, fala ou mesmo palestras.

O estudo do comportamento acústico do campo sonoro em ambientes fechados foi desenvolvido na tentativa de explicar objetivamente como se dá o comportamento acústico de ambientes, bem como os vários parâmetros necessários para descrever o comportamento do som dentro de tais ambientes. \cite{torres}.

O estudo da acústica de salas data do início do século XX, porém, com a evolução da tecnologia, a simulação acústica de salas sofreu um grande avanço, sendo possível realizar simulações que requerem cada vez menos tempo e gastos. Devido a esse avanço, atualmente é possível prever, por exemplo, o comportamento sonoro de salas ainda não construídas, simular a acústica de construções antigas não mais existentes e ainda melhorar a acústica de salas já existentes \cite{torres}.

Segundo \citeonline[pág.~1]{torres}, as possibilidades trazidas pela simulação acústica são muitas. Com o uso dessa técnica, por exemplo, é possível projetar/simular uma sala construída de tijolo e concreto com detalhes e com uma rica qualidade sonora, o qual seria relativamente complexo sem o uso da simulação acústica.

Neste Trabalho de Conclusão de Curso, será tratado um sistema de simulação acústica, sendo esse uma ferramenta de software livre baseada em uma abordagem multiagentes. Até o momento, não foi possível encontrar na literatura investigada uma ferramenta que reúne simulação acústica e uma abordagem multiagentes dentro do contexto de software livre.

\section{Questão de pesquisa}

A questão motivadora desta pesquisa concentra-se em:

\textit{É possível desenvolver um sistema que simule o comportamento do som dentro de um ambiente fechado utilizando uma abordagem multiagentes?}


\section{Justificativa}

A construção de salas, estúdios musicais, auditórios e ambientes, cuja propagação sonora é um fator importante, nem sempre é realizada de maneira adequada. A qualidade de som, a forma como este se propaga ao longo de uma sala e o isolamento acústico, são alguns dos fatores determinantes para um bom projeto. Adicionalmente, esses fatores podem influenciar diretamente na qualidade, nos custos e até mesmo no próprio comércio envolvido \cite{santos}. 

Em salas de aula, por exemplo, é importante saber se nas fileiras mais afastadas do professor é possível escutar com clareza as explicações. Além disso, é importante verificar se o projeto da sala foi eficiente o bastante para garantir que ruídos externos não interfiram na aula.

Já em estúdios musicais, é muito importante medir a qualidade do som, uma vez que esse fator está diretamente relacionado à sua atividade comercial. Além disso, verificar se o projeto atende ao requisito de isolamento sonoro do estúdio também requer especial atenção \cite{ramires}.

Tendo em vista as dimensões do auditório, é importante verificar qual o melhor posicionamento dos alto-falantes, para que todos dentro do auditório possam ouvir o som com a mesma qualidade.

Atualmente, uma das dificuldades dos desenvolvedores de projetos de construção civil é predizer se o som se comporta adequadamente dentro das especificações e necessidades deste projeto bem como verificar possíveis melhorias de projeto, incluindo dimensões, disposições, materiais, os quais atendam aos requisitos de qualidade do projeto \cite{santos}. Neste sentido, é proposto neste trabalho um software que apóie à visualização do comportamento dos elementos sonoros envolvidos em projetos de construção civil, auxiliando aqueles interessados na otimização acústica dentro de determinados ambientes.

\section{Objetivos}
\subsection{Objetivo geral}

Desenvolver um sistema que seja capaz de simular o comportamento do som dentro de um ambiente fechado, utilizando uma abordagem multiagentes, para que possa ser incorporado na implementação de novos simuladores acústicos, afim de potencializar o auxílio aos projetistas e/ou especialistas em acústica no que tange o acompanhamento e a avaliação dos parâmetros de seus projetos.

\subsection{Objetivos específicos}

\begin{enumerate}
\item Estudar o comportamento do som dentro de ambientes acústicos, identificando variáveis acústicas presentes dentro desses ambientes.

\item Identificar índices de absorção referentes aos materiais presentes nos ambientes em análise.

\item Propor um suporte tecnológico baseado em uma abordagem multiagentes, o qual será utilizado para a implementação da solução.

\item Explorar técnicas de programação, padrões de projeto e demais boas práticas da Engenharia de Software visando o desenvolvimento de um simulador manutenível e extensível.

\item Definir métricas de qualidade visando realizar a análise estática e a cobertura do código do simulador proposto, com base em uma abordagem de teste apropriada para o contexto, focada, principalmente, em testes de unidade.
\end{enumerate}

\section{Organização do documento}

Os próximos capítulos estão organizados da seguinte forma:

\begin{itemize}
	\item Referencial teórico - apresenta uma visão breve sobre os principais conceitos abordados neste trabalho, baseado na literatura disponível;
	\item Suporte tecnológico - apresenta as principais tecnologias utilizadas durante a condução deste trabalho.
	\item Metodologia - trata da descrição detalhada de todo o desenvolvimento do trabalho, tal como a explicação do tipo de pesquisa e a descrição da prova de conceito. 
	\item Resultados - apresenta os produtos de trabalho desenvolvidos, assim como os resultados dos testes realizados na validação dos mesmos.
	\item Conclusão - conclui o trabalho, apresentando as principais contribuições do mesmo bem como acorda possíveis trabalhos relacionados.
\end{itemize}