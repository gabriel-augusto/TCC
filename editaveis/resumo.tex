\begin{resumo}

Um projeto acústico deve ser cuidadosamente estudado, pois aspectos como dimensões e materiais aplicados influenciam diretamente no som do ambiente. O controle da acústica em um ambiente onde as dimensões já foram determinadas é realizado através da aplicação de materiais que influenciam no som. O emprego desses materiais deve ser cuidadosamente estudado a fim de aplicar materiais que contribuam com um bom projeto acústico e evitar o emprego de materiais supérfluos. O uso de simulações computacionais vem sendo cada vez mais frequente no dia a dia do projetista acústico, a fim de atingir projetos mais eficientes e que atendam às necessidades acústicas esperadas pelo cliente. Neste trabalho, será apresentado o projeto de um simulador que permite a simulação do comportamento do som dentro de um ambiente fechado. Esta simulação consiste em uma representação do som como uma partícula que percorre um determinado ambiente, identificando pontos de colisão e gerando as respectivas reflexões e absorções sonoras a fim de simular o evento de reverberação dentro do ambiente inserido. Este projeto visa a criação de uma máquina de raciocínio capaz de simular o comportamento do som dentro de um ambiente, permitindo então, o uso desta máquina na implementação de novos simuladores. Trata-se de uma ferramenta de software livre baseada no paradigma multiagentes, recorrendo a alguns recursos da orientação a objetos, onde seu principal foco é a comunidade de desenvolvimento de software livre. 

 \vspace{\onelineskip}
    
 \noindent
 \textbf{Palavras-chave}: simulador acústico. reflexão. absorção. reverberação. multiagentes.
\end{resumo}
